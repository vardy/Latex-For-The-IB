\section{What is \LaTeX?}
% Should be no more than half a page

\section{Installing \LaTeX}
There are a number of online \LaTeX{} editors. The most popular of them is \href{https://www.overleaf.com}{Overleaf}. Overleaf also has some useful \LaTeX{} \href{https://www.overleaf.com/learn}{documentation}.\par
% Add alternatives as I find out
There are also a number of OS-specific implementations detailed below.

\subsection{macOS and iOS}
The most popular standalone application for macOS and iOS is TexPad which costs approximately \SI{30}[\$]{} for the desktop version and \SI{15}[\$]{} for the mobile version. There is also Latex Presentation for presenting sideshows made with the Beemer document style. MathKey allows you to draw an equation and have the \LaTeX{} code appear automatically. There are a number of other less popular applications available as well.\par
It is also possible to download and the \LaTeX{} packages manually and compile .tex documents yourself. This allows you to edit those documents in any text editor. This further enables the production complex build script and other advanced features that will be covered in a later chapter. The macOS version is called \href{http://www.tug.org/mactex/}{MacTeX}. It can either be installed directly though their website or via HomeBrew. 

\subsection{Windows}
There are a number of \LaTeX{} editors available for Windows, the main ones are \href{https://miktex.org/}{MiKTeX}, \href{http://www.tug.org/protext/}{proTeXt}, and \href{http://www.tug.org/texlive/}{TeX Live}. TeX Live installs just the parser itself which simply takes in .tex documents and spits out pdfs. This allows you to edit those documents in any text editor. This further enables the production complex build script and other advanced features that will be covered in a later chapter.

\subsection{Linux}
Most Linux distributions have some form of \LaTeX{} package available in their package manager. This has different names in different names for each distribution. Note that these are command line tools that take .tex files and produce pdfs. This allows you to edit those documents in any text editor. This further enables the production complex build script and other advanced features that will be covered in a later chapter. Most Linux users are probably familiar with command line tools however if a GUI is desired it will be necessary to use an online editor.