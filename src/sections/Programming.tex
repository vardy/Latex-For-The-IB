\section{Programming and Advanced LaTeX}
% What does it mean to program in LaTeX

\subsection{Basic Macros}
There are a number of ways to define new macros. The most basic is the \verb=\newcommand= macro. This takes two arguments, the name of the new macro and what it expands to. Say I wanted to define a basic macro that expands to some text. The basic way to do this would be something like as follows. \verb=\newcommand{\hello}{Hello, World!}=. Because a macro is considered to be only a single value, the first pair of braces are actually not required though it can make the definition more confusing to read. If you were to remove those braces then the definition would look like \verb=\newcommand\hello{Hello, World!}=. This macro only allow defining a new macro, if you wish to redefine a preexisting macro, whether defined ourselves or by another package, a different command must be used which is \verb=\renewcommand=.
\subsubsection{Parameters}
Adding parameters to a macro is very easy. After the macro to define, just enter the number of parameters in brackets. To use the parameters, just use a number sign and then the number of the parameter. Here is an example of a macro using two parameters \verb=\newcommand{\hello}[2]{Hello #1, my name is #2.}=. This macro can then be used by typing \verb=\hello{Sarah}{Bryce}= with the result of \newcommand{\hello}[2]{Hello #1, my name is #2.}``\hello{Sarah}{Bryce}'' (quotation marks not included). There are more advanced ways of specifying parameters which is included in the \hyperref[section:programming/advancedFeatures/plainTeX]{Plain \TeX{}} section under \hyperref[section:programming/advancedFeatures]{Advanced features}.
\subsubsection{Optional Parameters}
\subsubsection{Macro Expansion Order}
\subsubsection{Dynamic Macro Expanding}
\subsubsection{Other Types of Macros}
\subsubsection{Internal Macro Definitions}

\subsection{Common Programming Features in LaTeX}
\subsubsection{Variables}
\subsubsection{Arrays}
\subsubsection{If Statements}
\subsubsection{Switch Statements}


\subsection{Advanced Features}\label{section:programming/advancedFeatures}
\subsubsection{Category Codes}
\subsubsection{Active Characters}
\subsubsection{Counters}
\subsubsection{Iteration}
\subsubsection{Computation}
\subsubsection{Working With Strings}
\subsubsection{Hooks}
\subsubsection{Command Line Parameters}
\subsubsection{Plain \TeX}\label{section:programming/advancedFeatures/plainTeX}


\subsection{Synthesising \LaTeX{} From Other Languages}


\subsection{Calling External Programs}


\subsection{Programming Tikz Pictures}
\subsubsection{Nodes}
\subsubsection{Links}
\subsubsection{Styles}
\subsubsection{Recursion}


\subsection{Writing Packages}


\subsection{Writing Classes}


\subsection{Breaking \LaTeX}
\subsubsection{Breaking Out of Environments}
\subsubsection{Changing Everything in The Middle of The Document}
\subsubsection{Obfuscation}