% Copyright 2018-2020 ImmortalPharaoh7, Bryce AS202313, Lynn Gu
%
% This file is part of Latex-For-The-IB.
%
% Latex-For-The-IB is free software: you can redistribute it and/or modify it
% under the terms of the GNU General Public License as published by
% the Free Software Foundation, either version 3 of the License, or
% (at your option) any later version.
%
% Latex-For-The-IB is distributed in the hope that it will be useful, but
% WITHOUT ANY WARRANTY; without even the implied warranty of
% MERCHANTABILITY or FITNESS FOR A PARTICULAR PURPOSE. See the GNU
% General Public License for more details.
%
% You should have received a copy of the GNU General Public License
% along with Latex-For-The-IB. If not, see http://www.gnu.org/licenses/.
%
\section{Setting up and Structuring}
\subsection{Useful packages}
Once you have your environment set up,
you're gonna have to add packages and configure things
in the preamble in order to respect the guidelines
(things like double spacing and whatnot).
Below is a example of what you generally will need in a preamble.
\begin{verbatim}
\documentclass[a4paper, 12pt]{article}

%Paragraph jumps and indentation
\setlength{\parskip}{1.6em}
\setlength{\parindent}{1.25cm}

%Border
\usepackage[left=1in, right=1in, top=1in, bottom=1in]{geometry}

%Double spacing
\usepackage{setspace}
\doublespacing

%Packages
\usepackage{amsmath}
\usepackage[dvipsnames]{xcolor}
\usepackage{mathtools}
\usepackage{amsfonts}
\usepackage{titlesec}

%Images
\usepackage{graphicx}
\graphicspath{ {./images/} }
\usepackage{wrapfig}
\usepackage{float}

%Tables
\usepackage{float}
\usepackage{multirow}
\usepackage{array}
\usepackage{tabu}

%Graphs
\usepackage{pgfplots}
\pgfplotsset{compat=1.5}

%Chemistry support
\usepackage{mhchem}

\titleformat{\section}
{\normalfont\large\bfseries}{\thesection}{1em}{}
\titleformat{\subsection}
{\normalfont\large\bfseries}{\thesubsection}{1em}{}

%Equation numbering
\counterwithin{equation}{section}

\usepackage{hyperref}
\urlstyle{same}
\end{verbatim}
Now a lot of those are quite straightforward and are just there to save time.
This is the point of this guide, to save you some time.
Packages related to tables and images will be talked about in later sections of this guide.

\verb|\usepackage[dvipsnames]{xcolor}| allows you to color your text,
which is generally useful for links since they aren't automatically colored.
More details over on \href{https://www.overleaf.com/learn/latex/Using_colours_in_LaTeX}{Overleaf's page}.

\verb|\usepackage{amsmath}| add maths symbols and environments like matrices,
while\\ \verb|\usepackage{mathtools}| adds tools to make things like long fractions aesthetically pleasing.
More details on mathtools are over on
\href{https://www.overleaf.com/learn/latex/Articles/Mathtools_-_for_beautiful_math}{Overleaf's page}.

\verb|\usepackage{titlesec}| allows you to configure the space between sections and subsections.
It's also used if you want to customize your sections or subsections.
We've used it in the example above (starting from \verb|\titleformat{\section}|)
to reduce the space after sections.
Since you have a hard 12 page limit in your G4 IAs,
might as well remove any unnecessary space before reducing word count.

\verb|\usepackage{hyperref}| allows you to add hyperlinks in your document,
whether those are for navigating your documents or to link websites.
We're using this right now to link you to
\href{https://www.overleaf.com/learn/latex/Hyperlinks}{Overleaf's documentation on hyperlinks}.

That's not all! There are other useful packages for G4 subjects,
which don't need to be in every preamble.

We recommend using \verb|\usepackage{siunitx}| when it comes to writing a physics IA,
since it provides a straightforward way to use SI and other units.
Which can be pretty handy.

We recommend using \verb|\usepackage[version=4]{mhchem}| for writing chemical reactions and formulas.
However, we (\href{https://www.overleaf.com/learn/latex/Chemistry_formulae}{and Overleaf})
recommend using \verb|\usepackage{chemfig}| for drawing molecules and other figures.

\subsection{Structuring your documents}
You can go on a couple of ways to structure your documents.
This part is just to make your code more aesthetically pleasing for you to look at it,
as well as being more organized.
This part just covers coding conventions which are arbitrary to some extent.

\begin{itemize}
\item Single file projects: Generally for IAs you'd like to write it as a single file if you wanna keep it simple.
Most of the times, there isn't really a reason to write multiple files unless you know what you're doing.
\item Multiple files projects: You can also write your IA / EE in multiple files, a file per section.
meaning that you will have a main file and is a bit more advanced when it comes to managing files and whatnot.
However, this strategy is better for EEs or IAs where you have a lot of graphs
and thus recompiling the whole file each time will be more time consuming.
Looking at the code of this guide might help you get an idea of what those projects look like.
Also you can check out \href{https://www.overleaf.com/learn/latex/Multi-file_LaTeX_projects}{Overleaf's guide}
on the matter for a lot more details which are targeted for very large \LaTeX{} projects.
\item Image folder: Generally you'd like to store your images in a folder
in order to note clutter up your main folder.
This is also why we have \verb|\graphicspath{ {./images/} }| in the boilerplate example.
\end{itemize}

There are also other conventions when it comes to writing \LaTeX{} code.
Although it's convenient to write a paragraph per line, provided that your IDE has appropriate text wrapping,
some people believe that manually breaking a line by pressing the enter key is useful,
since it doesn't really affect the pdf generated,
but it does help people who're looking at the github code, since they wouldn't have to scroll sideways 8 kilometers.
Keeping in mind that the IA and EE code aren't generally going to be published,
we're not going to recommend using a specific writing style;
but we believe it's better to pick up the good practices early on just in case.

