% Copyright 2018-2020 ImmortalPharaoh7, Lynn Gu
%
% This file is part of Latex-For-The-IB.
%
% Latex-For-The-IB is free software: you can redistribute it and/or modify it
% under the terms of the GNU General Public License as published by
% the Free Software Foundation, either version 3 of the License, or
% (at your option) any later version.
%
% Latex-For-The-IB is distributed in the hope that it will be useful, but
% WITHOUT ANY WARRANTY; without even the implied warranty of
% MERCHANTABILITY or FITNESS FOR A PARTICULAR PURPOSE. See the GNU
% General Public License for more details.
%
% You should have received a copy of the GNU General Public License
% along with Latex-For-The-IB. If not, see http://www.gnu.org/licenses/.
%
\section{Miscellaneous}
\subsection{French and Spanish support}
Now if you're a part of the small amount of people who do IB in either French or Spanish,
you can configure \LaTeX{} in order to adapt to the language environment.
There are things like transforming transforming the dot to commas when it comes to writing numbers,
changing the table names and whatnot. For more details
\href{https://www.overleaf.com/learn/latex/French}{here's the Overleaf link for French}
and \href{https://www.overleaf.com/learn/latex/Spanish}{this is for Spanish.}

This sums up what you need to put in the preamble but we've found out that (in French at least),
some things aren't implemented. For example when it comes to writing numbers in math mode,
the dot isn't turned into a comma. Yes you can commas in numbers, but there's this unnecessary space after the comma.
By putting 
\begin{verbatim}
\mathchardef\period=\mathcode`.
\DeclareMathSymbol{.}{\mathord}{letters}{"3B}
\end{verbatim}
in the preamble, the dots in the math mode numbers would automatically turn into commas.
Another solution is to put all your numbers in \verb|\nombre{}|.

Tables' captions are also left as ``Table'' (instead of Tableau), you can also put that code in the preamble to fix this
\begin{verbatim}
\addto\captionsfrench{\def\tablename{Tableau}}
\end{verbatim}

Last thing is you use \verb|\og \fg{}| instead of quotes in order to use the French quotes.
\subsection{CJK Support}
Since many IB students study Chinese, Japanese, or Korean, typesetting in these languages using \LaTeX{} may be a challenge because CJK characters are glued together rather than separated by spaces. CJK characters are supported by UTF-8 encoding; it is recommended by Overleaf to use compilers that directly support UTF-8, including XeLaTeX and LuaLaTeX. Detailed guides for typesetting \href{https://www.overleaf.com/learn/latex/Chinese}{Chinese}, \href{https://www.overleaf.com/learn/latex/Japanese}{Japanese}, and \href{https://www.overleaf.com/learn/latex/Korean}{Korean} are provided by Overleaf.
\subsection{Extended Essay Title Page}
Here's the boilerplate code for the EE title page:
\begin{verbatim}
\begin{titlepage}
    \begin{center}
        
        EE group\\
        Subject
        
        \vspace*{4cm}
 
        \textbf{Title:}\\
        Insert title here
        \vspace{1cm}

        \textbf{RQ:}\\
        Insert RQ here
        \vspace{4cm}


        Word count:
 
        \vfill
 
 
        \vspace{0.1cm}
 
 
       Year session
 
    \end{center}
\end{titlepage}
\end{verbatim}
We would like to note that this isn't the only way to write a title page,
but it's one way and you can change it to your liking if you want to.