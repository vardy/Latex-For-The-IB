% Copyright 2018-2020 Jarred Vardy, ImmortalPharaoh7, Bryce AS202313, Lenart Bucar
%
% This file is part of Latex-For-The-IB.
%
% Latex-For-The-IB is free software: you can redistribute it and/or modify it
% under the terms of the GNU General Public License as published by
% the Free Software Foundation, either version 3 of the License, or
% (at your option) any later version.
%
% Latex-For-The-IB is distributed in the hope that it will be useful, but
% WITHOUT ANY WARRANTY; without even the implied warranty of
% MERCHANTABILITY or FITNESS FOR A PARTICULAR PURPOSE. See the GNU
% General Public License for more details.
%
% You should have received a copy of the GNU General Public License
% along with Latex-For-The-IB. If not, see http://www.gnu.org/licenses/.
%
\section{Introduction}
\subsection{The objective of this guide}
Hello fellow reader, you're probably reading this guide wondering:
``What's the difference between this guide and any other \LaTeX{} guide?''.
Well for one, this guide isn't \LaTeX{} documentation;
it'll be a waste of time and effort to write documentation while there are a lot of good ones already.
We really recommend using \href{https://www.overleaf.com/learn/latex/Main_Page}{Overleaf's Documentation},
which is our favorite resource for learning \LaTeX{}.
This guide is tailored to IB students who have an idea on what \LaTeX{} is,
and would like to use it for their Internal Assessments or their Extended Essay.
This guide will mainly present the different tools generally used for writing your documents.
It will also contain boilerplate for things that might be a bit annoying to write,
as well as packages that are helpful for writing your documents and getting that 7 the \sout{best} \LaTeX{} way.

\subsection{Documentation and IDEs}
Before you go on, we assume that you already know basic \LaTeX{}.
If not, here's a list of good resources for learning:
\begin{itemize}
\item \href{https://www.overleaf.com/learn/latex/Main_Page}{Overleaf's Documentation}:
Honestly, best documentation ever.
You'll find it always useful and we will reference it a lot throughout this guide.
We highly recommend you using this.
\item \href{https://www.youtube.com/watch?v=VhmkLrOjLsw}{Derek Banas' Youtube video}:
If you don't know Derek Banas, here's what you have to know.
He's good at summarizing things, and this video covers a lot of \LaTeX{}'s documentation.
\item \href{https://ctan.org/}{CTAN}: It's not really that suitable for beginners but
it's the website you go to if you want to look up a package's official documentation.
Useful later on if you want to look for something specific.
\end{itemize}

To write \LaTeX{} documents, you're gonna need an IDE (Integrated Development Environment).
\href{https://beebom.com/best-latex-editors/}{Here}'s a link to a website that breaks down the most common used IDEs,
from where you can choose one you like or suits you.

\subsection{Contact us}
This guide is an ongoing project,
since the goal is to present everything an IB student would like to know before diving into writing their IAs/EE.
So if you have something you would like to suggest / add to this guide
or there are mistakes that you want to correct,
don't hesitate to contribute over on \href{https://github.com/ImmortalPharaoh7/Latex-For-The-IB}{our Github Repository}
or send a message to ImmortalPharaoh7\#7811 on Discord.